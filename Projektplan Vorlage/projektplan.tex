%================================================================================
%=============================== DOCUMENT SETUP =================================
%================================================================================

%\documentclass[lang=ngerman,inputenc=ansinew,fontsize=10pt]{ldvarticle}
\documentclass[lang=ngerman,inputenc=utf8,fontsize=10pt]{ldvarticle}
%PACKAGES

\usepackage{parskip}
\usepackage{subfigure}
\usepackage{ifthen}
\usepackage{comment}
\usepackage{color}	
\usepackage{colortbl}
\usepackage{soul}
\usepackage{tikz}
\usetikzlibrary{shapes,arrows}
\usepackage{tabularx}
\usepackage{todonotes}


\definecolor{lightgray}{rgb}{0.75,0.75,0.75}


%================================================================================
%================================= TITLE PAGE ===================================
%================================================================================

\title{Propellerman}
\subtitle{Projektpraktikum Informationsverarbeitung - Projektplan}
\author{Daniel Michalovics, Robin Kusterer, Tobias Maile}

\date{\today}

\begin{document}

\maketitle	
\thispagestyle{empty}
\vspace*{2cm}

\hrule

\section*{Motivation}

Neben vielen Vorlesungen, die sich mit der Theorie befassen, bietet das Praktikum Informationsverarbeitung endlich eine Möglichkeit das bisher erlangte Wissen in die Praxis umzusetzen. Die Aufgabe, ein autonom fliegendes Luftschiff selbstständig zu entwickeln, spornt uns an, da wir unsere eigenen Ideen umsetzen und präsentieren können. Darüber hinaus gewinnen wir durch das Projekt einiges an praktischer Erfahrung hinzu. Der Wettbewerb mit dem anderen Team motiviert uns ebenfalls, da wir natürlich versuchen werden unser Luftschiff mindestens genauso gut zu bauen.




\vspace*{1cm}
\hrule


\begin{figure}[!b]
\centering
\includegraphics[width=0.4\textwidth]{logo_kl.png}
\end{figure}

\newpage

%================================================================================
%================================= CONTENT ===================================
%================================================================================

\section{Beschreibung des Projekts}

\subsection*{Zielsetzung}
Das Ziel des Projekts ist es innerhalb von ca. 3 Monaten einen kleinen autonom fliegenden Zeppelin zu bauen. Dieser soll, ohne menschlichen Eingriff während des Flugs, einen willkürlich zusammengestellten Hindernisparcour durchfliegen und ein kleines Gewicht (Rettungspaket) an eine markierte Stelle transportieren. Der Start- und Endbereich, in dem der Flug startet bzw. endet sind vorgeschrieben. Die Positionen der Hindernisse sind vor Flugbeginn bekannt. Weitere Bedingungen sind, dass das Luftschiff während des gesamten Flugs nicht höher als 2 Meter fliegen darf und dass das Durchfliegen des Parcours nicht länger als 15 Minuten dauern darf.

\subsection*{Ansatz}
Der Zeppelin besteht aus einem, mit Helium gefüllten, Ballon und einer Gondel. Die Gondel ist im Prinzip eine Platte (z.B aus Holz, Styropor, Plastik) auf der folgende Komponenten montiert sind:
\begin{itemize}
\item \textbf{Arduino, Sensorplatine, Motortreiber:} zur geregelten Ansteuerung der Motoren und zur Kommunikation mit der Basisstation
\item \textbf{3 Motoren mit Propellern:}
\begin{itemize}
\item \textbf{in Flugrichtung:} für den Antrieb des Luftschiffs
\item \textbf{senkrecht zur Flugrichtung:} zum horizontalen Ausrichten des Luftschiffs
\item \textbf{zum Boden gerichtet:} für die Höhenregelung des Luftschiffs
\end{itemize}
\item \textbf{Elektromagnet:} als Vorrichtung zum Abwurf des Rettungspakets
\item \textbf{Akku:} für die Stromversorgung
\item \textbf{oben beschriebener Ballon:} für den Auftrieb
\end{itemize}

Für die Wegberechnung ist das Luftschiff durch drahtlose Telekommunikation mit einer Basisstation verbunden. Diese enthält die Koordinaten der zu fliegenden Strecke und sendet dem Luftschiff, abhängig von dessen aktueller Lage, die Position des nächsten Wegpunkts. Die Regelung selbst findet auf dem Arduino des Zeppelins statt.

\subsection*{Erfolgskriterien}
Damit das Projekt gelingt sollten folgende Kriterien erfüllt sein:
\begin{itemize}
\item \textbf{Die Gondel darf ein Gewicht von 100 Gramm nicht überschreiten.}
\item \textbf{Zeppelin und Basisstation müssen korrekte Daten miteinander austauschen.}
\item \textbf{Die Regelung des Flugs muss stabil und robust gegenüber Störungen, vor allem unerwarteten Winden, sein.}
\item \textbf{Alle Komponenten müssen fest genug montiert sein.}
\section{Meilensteine}
\end{itemize}


\begin{itemize}
\item \textbf{1. Projektplan bereit für Präsentation}
\begin{itemize}
\item Recherche bereits abgeschlossen
\item Liste aller benötigten Materialien erstellt
\item Lösungsansatz für Projektaufgabe gefunden
\item Projektstruktur vollständig
\item Präsentation erstellt
\end{itemize}
\item \textbf{2. Hardware vollständig und verfügbar}
\begin{itemize}
\item Sammelbestellung bereits geliefert
\end{itemize}
\item \textbf{3. Aktoren und Sensorentest}
\begin{itemize}
\item Aktoren steuerbar
\item Sensoren funktionieren (Testprogramme)
\item Ultraschall und Infrarot senden und empfangen möglich
\end{itemize}
\item \textbf{4. Gondel fertig gebaut:}
\begin{itemize}
\item Befestigung der Hardware
\item Gewichtsverteilung ideal gestalten
\item Verbindung vom Ballon zur Gondel
\item ansteuern der Gondel vom Boden aus
\item Verbindung IPS mit Gondel eingerichtet
\item Motoren sind montiert
\end{itemize}
\item \textbf{5.Erster manueller Flug}
\begin{itemize}
\item Luftschiff lässt ferngesteuerten Flug zu
\item Kommunikation zwischen Geräten funktioniert
\end{itemize}
\item \textbf{6.Softwareimplementierungen abgeschlossen:}
\begin{itemize}
\item Vorbereitungen auf autonomen Flug
\item IPS testen
\end{itemize}
\item \textbf{7. Simulation}
\begin{itemize}
\item Errechnen der Positionsfehler für Regelung
\item Anpassungen vornehmen
\end{itemize}
\item \textbf{8. Autonomer Flug}
\begin{itemize}
\item Ballon kommt durchs Ziel

\end{itemize}
\item \textbf{9. Dokumentation}
\begin{itemize}
\item Simultanes erfassen des aktuellen Fortschritts
\item abschließende schriftliche Darstellung des Projekts
\end{itemize}


\end{itemize}

\section{Projektstruktur (Arbeitspakete)}


\subsection*{Recherche/Materialbeschaffung}

Detailliertes studieren der Herangehensweisen früherer Gruppen.

Entwurf eines eigenen Konzeptes zur Bewältigung der Aufgabenstellung.

Fristgerechte Organisation passender Materialen.
 

\subsection*{Zusammenfügen der Hardware}

Anschließen einzelner Bauteile an den Computer, Wissenserwerb zum Umgang mit Software.\\
Alle Teammitglieder verschaffen sich einen Überblick über die Funktionsweise der gesamten mechanischen und elektronischen Ausrüstung, sodass jeder in der Lage ist eigenständig Änderungen durchzuführen.

Anschließend erfolgt die Montage der Gondel.

\subsection*{Programmierungsarbeiten}

Erstellen der einzelnen Programmteile (Streckenberechnung, Kommunikation, Regelung, IPS und Treiber zum Ansteuern der Gondel)

Anschließend optimieren und Zusammenfügen.



\subsection*{Dokumentation}
Während des gesamten Projekts wird der Fortschritt überprüft und festgehalten.

Nach dem Wettflug erfolgt die Ausarbeitung des Projekts.


\section{Aktivitäten- / Zeitplan}

\begin{center}
\begin{footnotesize}
\setlength{\arrayrulewidth}{1,05pt}
\begin{tabular}[htb]{|m{0,35\textwidth}|p{.05cm}|p{.05cm}|p{.05cm}|p{.05cm}|p{.05cm}|p{.05cm}|p{.05cm}|p{.05cm}|p{.05cm}|p{.05cm}|p{.05cm}|p{.05cm}|p{.05cm}|p{.05cm}|p{.05cm}|p{.05cm}|p{.05cm}|p{.05cm}|p{.05cm}|p{.05cm}|p{.05cm}|p{.05cm}|}
\hline
\textbf{Monat}& \multicolumn{4}{|c|}{Mai} & \multicolumn{5}{|c|}{Juni} & \multicolumn{4}{|c|}{Juli} \\
\hline
\textbf{Woche}&\tiny\textbf{18}&\tiny\textbf{19}&\tiny\textbf{20}&\tiny\textbf{21}& \tiny \textbf{22} & \tiny \textbf{23} & \tiny \textbf{24} & \tiny \textbf{25} & \tiny \textbf{26} & \tiny \textbf{27} & \tiny \textbf{28} & \tiny \textbf{29} & \tiny \textbf{30}\\
\hline
\hline
\rowcolor{lightgray} \textbf{Recherche und Materialbeschaffung}& \cellcolor{red} &\cellcolor{red} & & & & & & & & & & & \\
\hline
\rowcolor{lightgray} \textbf{Zusammenfügen der Hardware}& &\cellcolor{red} &\cellcolor{red} &\cellcolor{red} & & & & & & & & & \\
\hline
\rowcolor{lightgray} \textbf{Programmierungsarbeiten}& & &\cellcolor{red} &\cellcolor{red} &\cellcolor{red} &\cellcolor{red} &\cellcolor{red} &\cellcolor{red} &\cellcolor{red} &\cellcolor{red} & & & \\
\hline
\rowcolor{lightgray} \textbf{Päsentation}&\cellcolor{red} & & & &\cellcolor{red} & & & & & & & &\\
\hline
\rowcolor{lightgray} \textbf{Dokumentation}&\cellcolor{red} &\cellcolor{red} &\cellcolor{red} &\cellcolor{red} &\cellcolor{red} &\cellcolor{red} & \cellcolor{red} & \cellcolor{red} & \cellcolor{red} &\cellcolor{red} &\cellcolor{red} &\cellcolor{red} &\cellcolor{red}\\
\hline
\rowcolor{lightgray} \textbf{Meilensteine}&1 &2 &3 &4 &5 & & &6 &7 &8&9 & & \\
\hline

\end{tabular}
\end{footnotesize}
\end{center}


\section{Aufwandsabschätzung}

Für einen geregelten Arbeitsablauf müssen die Arbeitspakete sinnvoll auf-/ und eingeteilt werden.
Während auf der einen Seite die einzelnen Komponententests und die Software Programmierung sich aufteilen lässt, wird es auch zu Knotenpunkten kommen an denen bestimmte Aufgaben abgeschlossen sein müssen. Wichtige Knotenpunkte sind:
\begin{list}{-}{}

\item Zusammenbauen der Gondel.
\item Inbetriebnahme der Gondel.
\item Kommunikation der einzelnen Programme -> Bodenstation - Gondel - IPS.
\item Testen der Regelung am realen Objekt
\end{list}

Diese Punkte werden sehr kritisch behandelt, da sie einige Probleme direkt oder im späteren Verlauf bereiten können.


\section{Material- und Kostenplanung:}


Zwei wichtige Kriterien, welche bei jedem Bauteil beachtet werden müssen sind Preis und Gewicht, dabei müssen allerdings auch die speziellen Eigenschafften wie
\begin{list}{-}{}
\item Schub (Motoren/Propeller)
\item Laufzeit (Akku)
\item Stabilität (Gerüst)
…
\end{list}
eingehalten werden.

Die von Daedalus zur Verfügung gestellten Rohmaterialien sollten zum Erstellen des Gondelgerüstes ausreichen, und den Gewichtsanforderungen entsprechen. Neben der bereits vorhandenen Elektrik
\begin{list}{-}{}
\item Microkontroller
\item Sensorchip
\item Motorentreiber
\item Kabel, Isolierung, …
\end{list}
werden noch folgende Teile benötigt:

\begin{tabular}{|c|c|c|}
\hline
3x Elektromotoren & max 10 Euro & Conrad \\
\hline
min 1x Motorentreiber & etwa 10 Euro & sparkfun \\
\hline
Propeller & 1.50 - 8.00 Euro & Conrad \\
\hline
Akku & 9 - 30 Euro & Conrad \\
\hline
Kleber & etwa 10 Euro & Conrad \\
\hline
\end{tabular}

für den schlimmsten Fall ergäbe dies etwa eine Summe von 115 Euro.




\section{Risikoanalyse}

Risiken für das Projekt sind stets Verzögerungen und unerwartete Kosten. Dabei sind zu beachten:

\textbf{Inkompatible oder beschädigte Bauteile.}
\begin{list}{->}{}
\item Nachbestellungen können lange dauern und sind teuer.
\item alternative Teile verlangen möglicherweise Kompromisse und müssen neu getestet und verstanden werden.
\end{list}
\underline{Fazit:} frühe Zusammenstellung aller Komponenten und sorgfältiges Testen um Probleme früh zu erkennen.


\textbf{Programmfehler.}
\begin{list}{->}{}
\item Debugging ist sehr Zeitaufwendig
\item Fehler treten unerwartet auf und riskieren den Erfolg des Projektes
\end{list}

\underline{Fazit:} Alle Programme müssen gut vorbereitet werden, Arbeitsmittel müssen verstanden sein, enge Zusammenarbeit verhindert Probleme bei der Kommunikation der Programme/Programmteile.


\textbf{Ausfälle/ Terminprobleme.}
\begin{list}{->}{}
\item Besprechungen, Testphasen können nicht wahrgenommen werden
\item Kollaboration mit anderem Team/Tutoren kommt schwer zustande
\end{list}
\underline{Fazit:} frühe und verbindliche Terminplanung um den Arbeitsverlauf nicht zu verzögern.


\section{Änderungen des Projektablaufes}
Um Änderungen des Projektablaufs zu dokumentieren wird ein SpreadSheet in Google Drive erstellt, auf das alle Mitglieder lesenden und schreibenden Zugriff haben. Es handelt sich dabei um eine Tabelle mit 3 Spalten, die folgende Informationen enthält: Datum der Änderung, Die Änderung selbst, Eine kurze Begründung der Änderung.



\begin{tabular}[htbp]{|p{0,025\textwidth}||p{0,06\textwidth}|p{0,4\textwidth}|p{0,37\textwidth}|}
\hline
\textsc{\#} & \textsc{Datum} & \textsc{Änderung} & \textsc{Grund} \\
\hline
\hline
1 & & & \\[1em]
\hline
2 & & & \\[1em]
\hline
3 & & & \\[1em]
\hline
4 & & & \\[1em]
\hline
5 & & & \\[1em]
\hline
\end{tabular}

\end{document}
